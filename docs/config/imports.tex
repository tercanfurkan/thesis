\usepackage{graphicx}

\usepackage{enumitem}

%% Use this if you write hard core mathematics, these are usually needed
\usepackage{amsfonts,amssymb,amsbsy}

%% Use the macros in this package to change how the hyperref package below 
%% typesets its hypertext -- hyperlink colour, font, etc. See the package
%% documentation. It also defines the \url macro, so use the package when 
%% not using the hyperref package.

%\usepackage{url}

%% Use this if you want to get links and nice output. Works well with pdflatex.
\usepackage{hyperref}

\usepackage{subfig}
\captionsetup[table]{aboveskip=12pt}
\captionsetup[table]{belowskip=2pt}

\setlength{\tabcolsep}{6pt}
%\renewcommand{\arraystretch}{1.5} makes abstract table to jump to the next page

\setlength{\parindent}{0em}
\setlength{\parskip}{1em}

\hypersetup{pdfpagemode=UseNone, pdfstartview=FitH,
  colorlinks=true,urlcolor=red,linkcolor=blue,citecolor=black,
  pdftitle={Default Title, Modify},pdfauthor={Your Name},
  pdfkeywords={Modify keywords}}
  


\usepackage[
backend=biber,
style=numeric-comp,
sorting=ynt
]{biblatex}
\addbibresource{config/thesis.bib} %Imports bibliography file 

\usepackage[utf8]{inputenc}
\usepackage{glossaries}

\graphicspath{ {images/} }

\usepackage{csquotes}