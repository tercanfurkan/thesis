%%%%%%%%%%%%%%%%%%%%%%%%%%%%%%%%%%%%%%%%%%%%%%%%%%%%%%%%%%%%%%%%%%%%
%%%%%%%%%%%%%%%%%%%%%%%%%%%%%%%%%%%%%%%%%%%%%%%%%%%%%%%%%%%%%%%%%%%%
%%                                                                %%
%% An example for writting your thesis using LaTeX                %%
%% Original version by Luis Costa,  changes by Perttu Puska       %%
%% Support for Swedish added 15092014                             %%
%%                                                                %%
%% This example consists of the files                             %%
%%         thesistemplate.tex (versio 2.0)                        %%
%%         opinnaytepohja.tex (versio 2.0) (for text in Finnish)  %%
%%         aaltothesis.cls (versio 2.0)                           %%
%%         kuva1.eps                                              %%
%%         kuva2.eps                                              %%
%%         kuva1.pdf                                              %%
%%         kuva2.pdf                                              %%
%%                                                                %%
%%                                                                %%
%% Typeset either with                                            %%
%% latex:                                                         %%
%%             $ latex opinnaytepohja                             %%
%%             $ latex opinnaytepohja                             %%
%%                                                                %%
%%   Result is the file opinnayte.dvi, which                      %%
%%   is converted to ps format as follows:                        %%                         %%
%%                                                                %%
%%             $ dvips opinnaytepohja -o                          %%
%%                                                                %%
%%   and then to pdf as follows:                                  %%
%%                                                                %%
%%             $ ps2pdf opinnaytepohja.ps                         %%
%%                                                                %%
%% Or                                                             %%
%% pdflatex:                                                      %%
%%             $ pdflatex opinnaytepohja                          %%
%%             $ pdflatex opinnaytepohja                          %%
%%                                                                %%
%%   Result is the file opinnaytepohja.pdf                        %%
%%                                                                %%
%% Explanatory comments in this example begin with                %%
%% the characters %%, and changes that the user can make          %%
%% with the character %                                           %%
%%                                                                %%
%%%%%%%%%%%%%%%%%%%%%%%%%%%%%%%%%%%%%%%%%%%%%%%%%%%%%%%%%%%%%%%%%%%%
%%%%%%%%%%%%%%%%%%%%%%%%%%%%%%%%%%%%%%%%%%%%%%%%%%%%%%%%%%%%%%%%%%%%

%% Uncomment one of these, if you write in English:
%% the 1st when using pdflatex, which directly typesets your document in
%% pdf (use jpg or pdf figures), or
%% the 2nd when producing a ps file (use eps figures, don't use ps figures!).
\documentclass[english,12pt,a4paper,pdftex,sci,utf8]{aaltothesis}
%\documentclass[english,12pt,a4paper,dvips]{aaltothesis}

%% To the \documentclass above
%% specify your school: arts, biz, chem, elec, eng, sci
%% specify the character encoding scheme used by your editor: utf8, latin1

\usepackage{graphicx}

\usepackage{enumitem}

%% Use this if you write hard core mathematics, these are usually needed
\usepackage{amsfonts,amssymb,amsbsy}

%% Use the macros in this package to change how the hyperref package below 
%% typesets its hypertext -- hyperlink colour, font, etc. See the package
%% documentation. It also defines the \url macro, so use the package when 
%% not using the hyperref package.

%\usepackage{url}

%% Use this if you want to get links and nice output. Works well with pdflatex.
\usepackage{hyperref}

\usepackage{subfig}
\captionsetup[table]{aboveskip=12pt}
\captionsetup[table]{belowskip=2pt}

\setlength{\tabcolsep}{6pt}
\renewcommand{\arraystretch}{1.5}

\setlength{\parindent}{0em}
\setlength{\parskip}{1em}

\hypersetup{pdfpagemode=UseNone, pdfstartview=FitH,
  colorlinks=true,urlcolor=red,linkcolor=blue,citecolor=black,
  pdftitle={Default Title, Modify},pdfauthor={Your Name},
  pdfkeywords={Modify keywords}}
  


\usepackage[
backend=biber,
style=numeric-comp,
sorting=ynt
]{biblatex}
\addbibresource{thesis.bib} %Imports bibliography file 

\usepackage[utf8]{inputenc}
\usepackage{glossaries}
\makeglossaries
\newglossaryentry{RCS}
{
    name=RCS,
    description={Revision control system}
}
\newglossaryentry{LOC}
{
    name=LOC,
    description={Lines of code}
}


%% All that is printed on paper starts here
\begin{document}

%% Change the school field to specify your school if the automatically 
%% set name is wrong
%%
%% Korjaa vastaamaan korkeakouluasi, jos automaattisesti asetettu nimi on 
%% virheellinen 
% \university{aalto-yliopisto}
% \university{aalto University}
% \school{Sähkötekniikan korkeakoulu}
% \school{School of Electrical Engineering}

%% Only for B.Sc. thesis: Choose your degree programme. 
\degreeprogram{Service Design and Engineering}
%\degreeprogram{Elektroniikka ja sähkötekniikka}
%%

%% ONLY FOR M.Sc. AND LICENTIATE THESIS: Specify your department,
%% professorship and professorship code. 
\department{Department of Computer Science and Engineering}
%\department{Radiotieteen ja -tekniikan laitos}

%\professorship{Circuit theory}
%\professorship{Piiriteoria}
%\code{S-55}
%%

%% Valitse yksi näistä kolmesta
%%
%% Choose one of these:
%\univdegree{BSc}
\univdegree{MSc}
%\univdegree{Lic}

%% Oma nimi
%%
%% Should be self explanatory...
\author{Ömer Furkan Tercan}

%% Your thesis title comes here and again before a possible abstract in
%% Finnish or Swedish . If the title is very long and latex does an
%% unsatisfactory job of breaking the lines, you will have to force a
%% linebreak with the \\ control character. 
%% Do not hyphenate titles.

\thesistitle{Software Refactoring Rationale in Agile Development: A Case Study in a Lean Startup}

\place{Espoo}

%% For B.Sc. thesis use the date when you present your thesis. 
%% 
%% Kandidaatintyön päivämäärä on sen esityspäivämäärä! 
\date{31.03.2015}

%% B.Sc. or M.Sc. thesis supervisor 
%% Note the "\" after the comma. This forces the following space to be 
%% a normal interword space, not the space that starts a new sentence. 
%% This is done because the fullstop isn't the end of the sentence that
%% should be followed by a slightly longer space but is to be followed
%% by a regular space.
%%
%% Kandidaattiseminaarin vastuuopettaja tai diplomityön valvoja.
%% Huomaa tittelissä "\" -merkki pisteen jälkeen, 
%% ennen välilyöntiä ja seuraavaa merkkijonoa. 
%% Näin tehdään, koska kyseessä ei ole lauseen loppu, jonka jälkeen tulee 
%% hieman pidempi väli vaan halutaan tavallinen väli.
\supervisor{Assoc. Prof. Casper Lassenius} %{Prof.\ Pirjo Professori}

%% B.Sc. or M.Sc. thesis advisors(s). You can give upto two advisors in
%% this template. Check with your supervisor how many official advisors
%% you can have.
%%
%% Kandidaatintyön ohjaaja(t) tai diplomityön ohjaaja(t). Ohjaajia saa
%% olla korkeintaan kaksi.
%% 
%\advisor{Prof.\ Pirjo Professori}
%\advisor{D.Sc.\ (Tech.) Olli Ohjaaja}
%\advisor{M.Sc. Advisor}

%% Aalto logo: syntax:
%% Aaltologo: syntaksi:
%%
%% \uselogo{aaltoRed|aaltoBlue|aaltoYellow|aaltoGray|aaltoGrayScale}{?|!|''}
%%
%% Logo language is set to be the same as the document language.
%% Logon kieli on sama kuin dokumentin kieli
%%
\uselogo{aaltoRed}{''}

%% Create the coverpage
\makecoverpage


%% Note that when writting your master's thesis in English, place
%% the English abstract first followed by the possible Finnish abstract

%% English abstract.
%% All the information required in the abstract (your name, thesis title, etc.)
%% is used as specified above.
%% Specify keywords
\keywords{refactoring, code smells, anti-patterns, agile, lean}
%% Abstract text
\begin{abstractpage}[english]
What is refactoring? What is the goal of refactoring?

What is the role of software refactoring in agile development? What is the developer's perception of refactoring? What does the literature say about refactoring rationale? How does these to views match? What is the scope of this study?

How does this thesis conduct a study to answer the research questios? What is the research context in a sentence? What is the research methodology in summary?

What are the key findings? Limitations and future work?
\end{abstractpage}

%% Preface
%\mysection{Preface}
%\vspace{5cm}
%Otaniemi, 31.03.2013

%\vspace{5mm}
%{\hfill Ömer Furkan Tercan \hspace{1cm}}

\newpage


%% Table of contents. 
\thesistableofcontents

\newpage

%%Glossary - abbreviations
\printglossaries

\newpage

\listoftables


%% Corrects the page numbering, there is no need to change these
\cleardoublepage
\storeinipagenumber
\pagenumbering{arabic}
\setcounter{page}{1}


%% Text body begins. Note that since the text body
%% is mostly in Finnish the majority of comments are
%% also in Finnish after this point. There is no point in explaining
%% Finnish-language specific thesis conventions in English. Someday 
%% this text will possibly be translated to English.
\section{Introduction} \label{introduction}
In this chapter section \ref{background} presents the background and motivation of this study. Following section \ref{questions} introduces the objective and research questions. Finally, section \ref{structure} summarizes the structure and scope of this document. 
\subsection{Background and motivation} \label{background}
%What is refactoring \cite{fowlerRefactor}? Why and how critical is it for software development. What does the literature say about it's importance. If we scope it a bit down, what is the view of agile development to these questions?

%When should you refactor? When to add a new function, when to fix a bug, as you do a code review, 

%What is the tricky part when deciding on what and how much to refactor? Is this hard to define. Does it highly depend on the context? How credible is what the literature say about drivers on refactoring decisions. Is there a research gap to reveal empirical findings in order to contribute to the scientific knowledge regarding refactoring rationale?

%What would be the benefits, once we know more about refactoring rationale? What does the literature say about it's possible benefits? Take into account different viewpoints (customers, developers, etc.).

%Discuss the relevant literature. Have this specific topic been investigated? What are the touching points? Which areas of the topic will be overlapped (and why) in this study? What would be the additional contribution? Why does this contribution make sense?

%How do you claim to achieve this contribution? Summarize the research process. Why does it make sense to conduct the study in such a research context? 

Software systems are continuously forced to evolve as they can not resist change. During this process, software quality must be audited, secured, and maintained. Quality typically degenerates as a software is subjected to change during the course of its lifetime. Similarly, Lehman’s laws of software evolution state \cite{lehmanLaws}, a functionality increment of a system always brings a corresponding decrease in the quality and an increase in the internal complexity. Thus, maintaining such a system demands that in addition to adding new functionality, existing code must be continuously refactored, i.e., improved without adding functionality \cite{mantylaTaxonomy}. For instance, Microsoft reserves 20\% of its development effort to maintain and re-develop the code base of its products \cite{selbyMicrosoft}. 

Refactoring is the process of changing the internal structure of a software system without altering its external behavior \cite{fowlerRefactor}. Doing this will increase the understandability of code, make it easier to implement new features and debug the code. In most cases, refactoring can be achieved in terms of transforming the program into a better quality by fixing the quality defects like code smells, anti-patterns and other anomalies \cite{shatnawiEmpirical}.

Refactoring naturally fits in the process of software reengineering \cite{demeyerReengineering}, which aims to restructure legacy software. Therefore, refactoring as a concept was previously practiced within software restructuring. As object-orented design emerged as a contemporary concept, the term refactoring surfaced in the early nineties \cite{fowlerRefactor}. 

Researchers contributed large size of knowledge and concepts in the area of software refactoring. Fowler and Beck introduced 22 problematic code structures that requires refactoring \cite{fowlerRefactor}. Mäntylä et al. proposed a classification \cite{mantylaTaxonomy} for the 22 code smells introduced by Fowler and Beck and further investigates the correlation between the smells through an initial empirical study to help understand how different smells are connected to each other. Brown et al. explains development anti-patterns \cite{brownAntiPatterns}, which have similarities with code smells.

Researchers have also contributed with surveys and empirical studies. Siy and Votta \cite{siyInspection} studied data of 130 code inspection sessions and came up with four groups of code maintenance drivers, namely documentation, style, portability, and safety. Further, they present empirical evidence on how developers perceive the discovered drivers. In their survey \cite{yamashitaDevelopers}, Yamashita and Moonen inspects code smells from the developers perspective and presents a prioritized list of smells. While also having empirical parts, there are also studies introducing automated tool support for detecting specific refactoring candidates \cite{simonMetrics,kataokaAutomated,tourweRefactoring}. 

However, there is little evidence to justify the use of code smell in refactoring. In particular, the study of the human perception of what is a code smell and how to deal with it has been mostly neglected in the past. Therefore, further research on refactoring can bring a significant contribution if this research gap is considered. In particular a research, which 1) considers the human factorwhen determining the drivers behind refactoring decisions, and 2) identifies the refactoring solutions practices by developers in addressing different refactoring drivers. 

Motivated by the findings in earlier studies, this thesis investigates the refactoring rationale in a lean startup ICT company. Attention has been paid to understand the drivers behind refactoring decisions in the company's development process. The results of the case study is analyzed, discussed and the most significant findings are compared with the scientific knowledge regarding software refactoring. We will also discuss the issues which needs more thorough investigation and research in the future.

In addition, Chapter \ref{literature}, presents an overview of software refactoring. It descibes refactoring reasons found in literature, in terms of code smells and anti-patterns. Finally, it introduces survey and empirical studies on the relation between refactoring reasons and how they are perceived by software professionals.

%\subsection{Scope of the thesis} \label{scope}

%This study presents empirical findings on refactoring decisions for a two months period in a lean startup ICT company. Therefore, the findings are limited to the case company's agile development processes.

%Data was collected from the company's version control system during the study period. Historical data was scoped out due to the time limitations of this study. 

\subsection{Objective and research questions} \label{questions}
According to the motivation stated in section \ref{background}, there is a need to investigate and reveal the correlation between how literature have conceptualized software refactoring rationale and how it is actually percieved by developers. Accordingly, an empirical study can be conducted investigating refactoring decisions in a case company. The empirical findings can then be further analyzed and compared with the scientific knowledge regarding software refactoring rationale. 

This motivation leads to the main study objective:

\textbf{Objective:} Study refactoring rationale, particularly how it is perceived by the scientific knowledge versus agile software developers, to identify neglected threats to software evolution.

The following research questions were derived based on the main objective:

\begin{enumerate}[label=\textbf{RQ\arabic*}]
\item What does literature say about refactoring and when it is needed?

This research question tries to investigate relevant literature to form a basis on refactoring, in particular to present a refactoring rationale taxonomy based on related work.
\item How does developers perceive refactoring rationale?

This research question is of explorative nature and it tries to find empirical evidence on refactoring rational, taking into account the human factor. 
\item What does the correlation of the two reveal in terms of threats to software evolution?

This research question aims to discuss how well the developer's perception fits to the presented taxonomy. Based on this discussion, it aims to identify neglected threats to software evolution.
\end{enumerate}

\subsection{Structure and scope of the thesis} \label{structure}
The summary of the scope and structure of the thesis is presented out in the folowing.

\subsubsection*{Introduction}
The background and motivation of the thesis is presented, accompanied by the main objection and research questions. 

\subsubsection*{Software refactoring}
Overview of the literature on software refactoring is presented. Refactoring reasons found in literature, in terms of code smells and anti-patterns is described, accompanied by survey and empirical studies on the relation between refactoring reasons and how they are perceived by software professionals.


\subsubsection*{Research methodology}
The main objective and research questions are elaborated. Relation between research questions and conducted studies are presented. In order to explain the case study environment, case company and their processes are intoduced. Finally, data collection and analysis methods are explained.

\subsubsection*{Case study findings and discussion}
Case study findings are tightly coupled with the the literature findings. First, as a result of the conducted case study, collected and analysed empirical data on refactoring decisions are presented. Secondly, the findings are compared and discussed together with the literature findings in tho parts. 1) The correlation between findings and refactoring reasons found in literature is identified. 2) The relation between our empirical findings and empirical results in the literature is discussed. 


\subsubsection*{Conclusions}
This chapter concludes the outcome of the thesis and also presents potential threats to validity and future research areas. 

\clearpage
\clearpage

\section{Literature study} \label{literature}
In addition to the references mentioned in the sub-sections, take a look at the following papers:
\begin{itemize}
\item Yamashita and Leon (2012)
\item Yamashita and Leon (2013)
\item Mäntylä (2009)
\item Arcoverde et al. (2011)
\item Mäntylä (2005)
\item Mäntylä and Lassenius (2006)
\end{itemize}

\subsection{Software refactoring}
Define refactoring referring mainly to Brown et al. (1998) and Fowler (2000). 

Explain the importance and benefits of refactoring referring to at least Khomh et al. (2009), Cusumano et al. (1997), Cusumano et al. (1997),  

\subsection{Refactoring drivers}
Admit the basis of the presented taxonomy comes from Fowler (2000) but it is inspired (or taken from) Mäntylä et al. (2003). 

Attempt to discuss and improve the taxonomy with at least Brown et al. (1998), Mäntylä and Lassenius (2006), Yamashita et al. (2013)
\clearpage

\section{Research setting} \label{research setting}
The main objective of this study is to gain an understanding on refactoring rationale, in particular how it is described in the relevant literature versus how it is perceived by agile software developers. Thus, identify neglected threats to software evolution. 

The objective is targeted in terms of a) studying relevant literature to form a basis on refactoring, particularly on refactoring rationale, b) finding empirical evidence on how refactoring rationale is percieved by developers, and c) discussing the inclusionary nature of the relevant literature study and our empirical findings. The relation of these studies to the research questions is presented in Table \ref{table:researchQuestions}.

\begin{table}[ht!]
\centering
\caption{Relation between studies and research questions}
\label{table:researchQuestions}
    \begin{tabular}{ p{10cm} p{2cm} p{2cm}  }
     \hline
     \textbf{Research question}  &\textbf{Literature}  &\textbf{Empirical}  \\
     \hline
     RQ1: What does literature say about refactoring and when it is needed?                         &X      &-  \\
     \hline
     RQ2: How does developers perceive refactoring rationale?                                       &-      &X  \\
     \hline
     RQ3: What does the correlation of the two reveal in terms of threats to software evolution?    &x      &x  \\
     \hline
    \end{tabular}
\centering
\caption*{  \textbf{Legend}: "X" indicates higher emphasis then "x", while "-" indicates no emphasis}
\end{table}

Finding an answer to RQ1 requires a literature study. This research question aims to present a software refactoring rationale taxonomy, which will act as the first building block when answering RQ3.

Finding an answer to RQ2 through a real world investigation overlaps with Yin's (2003) definition of a case study --- an empirical method aimed at investigating contemporary phenomena in their context. Similarly, the findings will act as the second building block when answering RQ3.

RQ3 aims to study the correlation between RQ1 and RQ2. In addition, it conducts a minor literature study on software evolution, to reveal neglected threats by taking the studied correlation into account.

\subsection{Case company} \label{caseCompany}
The case study was conducted in a lean start-up ICT company based in Helsinki, Finland. The company is producing  telecommunication application services enabling video calling from multiple endpoints, including web browsers, smart TV's and mobile devices. 
\subsubsection*{Software under study} \label{software}
The service platform consists of a server backend and multiple client endpoints. However, the scope of this study includes the server backend and two of the clients, since only these three components were actively developed within the data collection period. 

\begin{enumerate}[label=\textbf{C\arabic*}]
\item The server component presented in Table \ref{table:serverLoc}, was implemented using javascript, having nodejs as the runtime environment.

\item An android client component presented in Table \ref{table:androidLoc}, was implemented using java.
    
\item A web client component presented in Table \ref{table:webLoc}, was implemented using javascript, having angularjs as the web application framework.
\end{enumerate}

\begin{table}[ht!]
\centering
\caption{Server component; number of files and \gls{LOC}}
\label{table:serverLoc}
    \begin{tabular}{ |p{3cm} p{2cm} p{2cm} p{2cm} p{2cm}|  }
     \hline
     Language       &Files  &Blank  &Comment    &Code\\
     \hline
     \hline
     Javascript     &36     &582    &131        &4059\\
     Json           &4      &1      &0          &251\\
     Python         &5      &72     &46         &213\\
     Html           &1      &0      &1          &43\\
     Shell          &2      &21     &5          &39\\
     \hline\hline
     Sum            &48     &676    &183        &4605\\
     \hline
    \end{tabular}
\end{table}

\begin{table}[ht!]
\centering
\caption{Android client component; number of files and \gls{LOC}}
\label{table:androidLoc}
    \begin{tabular}{ |p{3cm} p{2cm} p{2cm} p{2cm} p{2cm}|  }
     \hline
     Language       &Files  &Blank  &Comment    &Code\\
     \hline\hline
     Xml            &707    &2687   &4921       &25950\\
     Java           &143    &2699   &5873       &15078\\
     Shell          &2      &50     &21         &237\\
     \hline\hline
     Sum            &852    &5436   &10815      &41265\\
     \hline
    \end{tabular}
\end{table}

\begin{table}[ht!]
\centering
\caption{Web client component; number of files and \gls{LOC}}
\label{table:webLoc}
    \begin{tabular}{ |p{3cm} p{2cm} p{2cm} p{2cm} p{2cm}|  }
     \hline
     Language&Files&Blank&Comment&Code\\
     \hline\hline
     Javascript   & 37    &542&   161&3631\\
     Less&   15  & 275   &10 & 1359\\
     Html &24 & 58&  17&540\\
     Json    &2 & 0&  0 & 0\\
     \hline\hline
     Sum&78&875&188&5606\\
     \hline
    \end{tabular}
\end{table}
\subsubsection*{Development process} \label{process}
The organization is a lean startup, following lean development flow principles and employing most of the agile development practices. Practices worth mentioning includes; continuous integration and deployment, continuous improvement (process and software quality), efficient and face-to-face communication, peer reviews, one peace flow, pair programming, testing as an integral part of development, and collective code ownership.

As it is a key principle of lean development, in ensuring continuous integration and deployment, continuous software quality improvement takes place routinely but critically within the organization. In particular, software refactoring serves as the main approach in this context.  

The empirical study was introduced to the developers during continuous process improvement meetings, named as retrospectives. As part of the study, developers were requested to put addition attention in reporting their refactoring related tasks. Collective code ownership and peer reviews were routine practices at present. Therefore, developers were already used to report their daily tasks using the source-code revision control system \gls{RCS}. 

\subsubsection*{Developers} \label{developers}
All developers of the organization participated in the empirical study. During this period, 6 developers reported their refactoring tasks in the source-code \gls{RCS}.

<<A figure presenting developers and their work experience levels>>

\subsection{Data collection} \label{data collection}
Data was gathered from a single source. The single source of information was stored in the source-code \gls{RCS}. Developers reported their refactoring related tasks to the source-code \gls{RCS}. Since collective code ownership and peer reviews where routine practices at present, this extra duty required almost no extra effort from developers. 

Prior to the empirical data collection kick-off, several group discussions were conducted to motivate developers to highlight their refactoring tasks in their notes and to place useful conventions for data collection and analysis. In order to ease the data collection and analysis process, a convention of labeling their reports was introduced to developers.

Developers were using git as the source-code \gls{RCS}. Therefore, developers highlighted their refactoring tasks using git commit messages and git code line comments. Subsequently, a public API offered by the cloud-based git repository service was used to retrieve the commit messages, commit changes, and code line comments. The API offered json as a response format. Therefore, this made it trivial to collect, filter, further analyze and categorize refactoring tasks based on developer notes.  

In the cases were developer notes lacked sufficient amount of detail on the rational behind a refactoring decision, informal communication was used to resolve the ambiguity. Thanks to a) peer reviews with having short feedback cycles, b) efficient and face-to-face communication possibility, and c) technical reviews following short iterations, ambiguous developer notes were minimized. 

\subsection{Data analysis} \label{data analysis}
How was the data filtered, cleaned, and categorized? What tools and analysis methods were used?
\clearpage

\section{Case study findings} \label{case study}
Present findings based on data collection and analysis. Discuss correlation with the literature findings.
\subsection{Identified refactoring rationales}
\subsubsection*{Algorithm change}
Improve code algorithm, remove excessive nesting, simplify complexity or add/remove additional parameters/libraries to improve understandability and robustness

\subsubsection*{Code formatting}
Improve code format – remove/add blank lines, modify indentationl modify layout -  for better readability

\subsubsection*{Comments}
Add, remove or update line, method, class, general comments and pseudocode to improve understandability

\subsubsection*{Debugging}
Add, remove, update debugging related comments and logs, or any other improvements to help development

\subsubsection*{Decoupling}
Decouple tightly coupled classes by applying unidirectional associations, replacing inheritance with delegation to decrease intimacy or move method to correct class to limit feature envy, eventually achieve better decoupling and improved modularity

\subsubsection*{Delegate lazy class}
Delegate lazy class or method behavior to remove middle man or to remove dispensable code

\subsubsection*{Design inheritance}
improve inheritance misuse by making proper use of existing inheritance hierarchies, or using common interfaces to limit change Preventers, duplication and improve maintainability 

\subsubsection*{DevOps}
Any code improvement or configuration to help development and operational tasks

\subsubsection*{Encapsulation}
Redefine the scope of the class behavior and improve access rights

\subsubsection*{Extract method or class}
Divide large class or long method for readability and modularity

\subsubsection*{Modify code hierarchy}improve class, method, conditional structure by grouping code or other minor structure Changes to improve readability

\subsubsection*{Name change}
Changing variable, method, class, file, package naming to improve readability

\subsubsection*{Performance optimization}
Improve code algorithm and reduce computational complexity or add/remove additional parameters/libraries to improve performance

\subsubsection*{Remove dead code}
Remove any unused lines remaining in code due to legacy reasons, evolving software, changing functionality and refactoring

\subsubsection*{Replace data clumps with objects}
Wrap related data clumps into an object, use object orientation properly to reduce data clumps, long parameter list, large classes, long methods, primitive obsession and handle code more effectively 

\subsubsection*{Simplify speculative complexity}
Simplify code based on speculative generality, misunderstood requirements and over design

\subsubsection*{Unify code}
Create utility function based on duplicated code, generalize existing method behavior and serve similar requests, or use constants to remove duplicated code, limit change preventers and improve maintainability

\subsection{Discussion}
\clearpage

\section{Discussions} \label{discussions}
Discuss correlation between RQ1 and RQ2. Correlation shall reveal findings on neglected threats to software evolution
\clearpage

\section{Conclusions} \label{conclusions}
\subsection{Threads to Validity} \label{validity}
\subsection{Future Work} \label{future}
\clearpage

\section{Temporary references}
\begin{itemize}
\item Arcoverde, Roberta, Alessandro Garcia, and Eduardo Figueiredo. "Understanding the longevity of code smells: preliminary results of an explanatory survey." Proceedings of the 4th Workshop on Refactoring Tools. ACM, 2011.

\item Cusumano, Michael A., and Richard W. Selby. "Microsoft secrets." (1997).

\item Cusumano, Michael, and David Yoffie. "Competing on Internet time." (1998).

\item Fowler, Martin. "Refactoring: Improving the Design of Existing Code. 2000."

\item Khomh, Foutse, Massimiliano Di Penta, and Y. Gueheneuc. "An exploratory study of the impact of code smells on software change-proneness." Reverse Engineering, 2009. WCRE'09. 16th Working Conference on. IEEE, 2009.

\item Lehman, Meir M. "Programs, life cycles, and laws of software evolution." Proceedings of the IEEE 68.9 (1980): 1060-1076.

\item McCormick, Hays W., Thomas J. Mowbray, and Raphael C. Malveau. "AntiPatterns: refactoring software, architectures, and projects in crisis." (1998).

\item Mäntylä, Mika V., Jari Vanhanen, and Casper Lassenius. "A taxonomy and an initial empirical study of bad smells in code." Software Maintenance, 2003. ICSM 2003. Proceedings. International Conference on. IEEE, 2003.

\item Mäntylä, Mika V. "An experiment on subjective evolvability evaluation of object-oriented software: explaining factors and interrater agreement." Empirical Software Engineering, 2005. 2005 International Symposium on. IEEE, 2005.

\item Mäntylä, Mika V., and Casper Lassenius. "Drivers for software refactoring decisions." Proceedings of the 2006 ACM/IEEE international symposium on Empirical software engineering. ACM, 2006.

\item Mäntylä, Mika V., and Casper Lassenius. "Subjective evaluation of software evolvability using code smells: An empirical study." Empirical Software Engineering 11.3 (2006): 395-431.

\item Mäntylä, Mika. "Software evolvability-empirically discovered evolvability issues and human evaluations." (2009).

\item Yamashita, Aiko, and Leon Moonen. "Do developers care about code smells? An exploratory survey." Reverse Engineering (WCRE), 2013 20th Working Conference on. IEEE, 2013.

\item Yamashita, Aiko, and Leon Moonen. "Do code smells reflect important maintainability aspects?." Software Maintenance (ICSM), 2012 28th IEEE International Conference on. IEEE, 2012.

\item Yamashita, Aiko, and Leon Moonen. "Exploring the impact of inter-smell relations on software maintainability: an empirical study." Proceedings of the 2013 International Conference on Software Engineering. IEEE Press, 2013.

\item Yin, Robert K. "Applications of case study research (applied social research methods)." Series, 4th. Thousand Oaks: Sage Publications (2003).
\end{itemize}

\clearpage

\printbibliography[
heading=bibintoc,
title={References}
]
%% The \phantomsection command is nessesary for hyperref to jump to the 
%% correct page, in other words it puts a hyper marker on the page.

%\phantomsection
%\addcontentsline{toc}{section}{\refname}
%\addcontentsline{toc}{section}{References}
%\begin{thebibliography}{99}

%\bibitem{Kauranen} Kauranen,\ I., Mustakallio,\ M. ja Palmgren,\ V.
%  \textit{Tutkimusraportin kirjoittamisen opas opinn\"aytety\"on
%    tekij\"oille.}  Espoo, Teknillinen korkeakoulu, 2006.

%\bibitem{Itkonen} Itkonen,\ M. \textit{Typografian k\"asikirja.} 3.\
%  painos.  Helsinki, RPS-yhti\"ot, 2007.

%\bibitem{Koblitz} Koblitz,\ N. \textit{A Course in Number Theory and
%    Cryptography. Graduate Texts in Mathematics 114.}  2.\ painos. New
%  York, Springer, 1994.
  
%\bibitem{bcs} Bardeen,\ J., Cooper,\ L.\ N. ja Schrieffer,\ J.\ R.
%  Theory of Superconductivity. \textit{Physical Review,} 1957, vol.\
%  108, nro~5, s.\ 1175--1204.

%\bibitem{viittaaminen} Kilpel\"ainen,\ P. WWW-l\"ahteisiin viittaaminen
%  tutkielmatekstiss\"a. Verkkodokumentti. P\"aivitetty 26.11.2001.
%  Viitattu 19.1.2007. Saatavissa:
%  \url{http://www.cs.uku.fi/~kilpelai/wwwlahteet.html.}

%\end{thebibliography}

\end{document}
