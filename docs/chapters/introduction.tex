In this chapter section \ref{background} presents the background and motivation of this study. Following section \ref{scope} presents the scope of the thesis. Section \ref{questions} introduces the objective and research questions. Finally, section \ref{structure} summarizes the structure of this document. 
\subsection{Background and motivation} \label{background}
What is refactoring \cite{fowlerRefactor}? Why and how critical is it for software development. What does the literature say about it's importance. If we scope it a bit down, what is the view of agile development to these questions?

Refactoring is the process of making small code level changes to improve its internal structure. The cumulative effect of these small changes can radically improve software design \cite{fowlerRefactor}. With refactoring the software design is formed continuously during development rather then up front. 

Refactoring is performed to make the software more readable. Also it can be applied as performance optimizations. Refactoring does not change the observable behavior of the software. Software still carries out the same function that it did before. Any user, whether an end user or another programmer, can not tell that things have changed \cite{fowlerRefactor}.  

Then why refactor? Refactoring improves the design of the software, makes software easier to understand, helps to find bugs, helps to program faster, 

When should you refactor? When to add a new function, when to fix a bug, as you do a code review, 

What is the tricky part when deciding on what and how much to refactor? Is this hard to define. Does it highly depend on the context? How credible is what the literature say about drivers on refactoring decisions. Is there a research gap to reveal empirical findings in order to contribute to the scientific knowledge regarding refactoring rationale?

What would be the benefits, once we know more about refactoring rationale? What does the literature say about it's possible benefits? Take into account different viewpoints (customers, developers, etc.).

Discuss the relevant literature. Have this specific topic been investigated? What are the touching points? Which areas of the topic will be overlapped (and why) in this study? What would be the additional contribution? Why does this contribution make sense?

How do you claim to achieve this contribution? Summarize the research process. Why does it make sense to conduct the study in such a research context? 

Motivated by the findings in earlier studies, this thesis investigates the refactoring rationale in a lean startup ICT company. Attention has been paid to understand the drivers behind refactoring decisions in the company's agile development process. The results of the case study is analyzed, discussed and the most significant findings are compared with the scientific knowledge regarding software refactoring rationale. We will also discuss the issues which needs more thorough investigation and research in the future.

This study also contains a literature review chapter. In chapter \ref{literature}, we will discuss software refactoring and refactoring rationale as it is described in the literature. Then, we will explore refactoring drivers defined by the scientific knowledge, including anti-patterns and code smells.

\subsection{Scope of the thesis} \label{scope}

\subsection{Objective and research questions} \label{questions}
According to the motivation stated in section \ref{background}, there is a need to investigate and reveal the correlation between how literature have conceptualized software refactoring rationale and how it is actually percieved by developers. Accordingly, an empirical study can be conducted investigating refactoring decisions in a case company. The empirical findings can then be further analyzed and compared with the scientific knowledge regarding software refactoring rationale. 

This motivation leads to the main study objective:

\textbf{Objective:} Study refactoring rationale, particularly how it is perceived by the scientific knowledge versus agile software developers, to identify neglected threats to software evolution.

The following research questions were derived based on the main objective:

\begin{enumerate}[label=\textbf{RQ\arabic*}]
\item What does literature say about refactoring and when it is needed?

This research question tries to investigate relevant literature to form a basis on refactoring, in particular to present a refactoring rationale taxonomy based on related work.
\item How does developers perceive refactoring rationale?

This research question is of explorative nature and it tries to find empirical evidence on refactoring rational, taking into account the human factor. 
\item What does the correlation of the two reveal in terms of threats to software evolution?

This research question aims to discuss how well the developer's perception fits to the presented taxonomy. Based on this discussion, it aims to identify neglected threats to software evolution.
\end{enumerate}

\subsection{Structure of the thesis} \label{structure}
%% In a thesis, every section starts a new page, hence \clearpage